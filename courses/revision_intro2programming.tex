\documentclass[12pt,a4paper]{article}

\usepackage[francais]{babel}
\usepackage[utf8]{inputenc}
\usepackage{epigraph}
\usepackage[T1]{fontenc}
\usepackage{listings}
\usepackage{graphicx}
\usepackage{parskip}
\usepackage{xcolor}


%% config
\definecolor{deepblue}{rgb}{0,0,0.5}

\graphicspath{{images/}}
\lstset{
    language={Python},
    basicstyle=\ttfamily\small, 
    tabsize=4,
    keywordstyle=\bold,
    commentstyle=\color{gray},
    backgroundcolor=\color{lightgray},
    otherkeywords={self},
	keywordstyle=\ttfamily\small\color{deepblue},
    frame=single
    showtabs=false,
    showspaces=false,
    showstringspaces=false,
    inputencoding=utf8,
    literate={à}{{\`a}}1 {è}{{\`e}}1,
}

%%macro

\newcommand{\path}[1]{\texttt{#1}}

\newcommand{\codeintext}[1]{\texttt{#1}}
\newcommand{\response}[0]{Réponse:\hrulefill\\\\}


\begin{document}

\section*{Exercices d'échauffement}

\subsection*{Exercice 1}

Trouvez ce que va afficher l'instruction \codeintext{print} dans les scripts suivants:

\begin{enumerate}

% 1
\item
\begin{lstlisting}
a = 2
b = 3
a = a + b
c = a * 2
print(c)
\end{lstlisting}
\response % 10

% 2
\item
\begin{lstlisting}
i = 10
i += 2
j = i * 3

print(i + j)
\end{lstlisting}
\response % 48

% 3
\item
\begin{lstlisting}
frog = "gama"
letter = "m"
hulk_ray =  frog[:2] + letter + frog[2:]

print(hulk_ray)
\end{lstlisting}
\response

\pagebreak
% 4
\item
\begin{lstlisting}
vowel = ['a', 'e', 'i', 'o', 'u', 'y']

word = 'Boum!'
new_word = ''

for letter in word:
	multi = 1
	if letter in vowel:
		multi = 3
	for index in range(multi):
		new_word += letter

print(new_word)
\end{lstlisting}
\response % Booouuum!

% 5
\item
\begin{lstlisting}

monster_damage = 3
hero_damage = 2
hero_hp = 20
monster_hp = 15

while(hero_hp > 0 and monster_hp > 0):
	hero_hp -= monster_damage
	monster_hp -= hero_damage

if hero_hp <= 0 and monster_hp > 0:
	print("Monster wins!")
elif hero_hp > 0 and monster_hp <= 0:
	print("Hero wins!")
else:
	print("Double KO!")
\end{lstlisting}
\response % Monster wins!"
\pagebreak
% 6
\item
\begin{lstlisting}

def bubble(array):
	move = True
	
	while move:
		move = False
			
		for index in range(len(array)-1):
			if array[index] > array[index + 1]:
				glass = array[index]
				array[index] = array[index + 1]
				array[index + 1] = glass
				move = True
	return array
	
l = [3, 1, 4 , 1, 6]
print(bubble(l))

\end{lstlisting}
\response % [1, 1, 3 , 4, 6]
\end{enumerate}
\pagebreak
\subsection*{Exercice 2}

Ecrivez la fonction \codeintext{do\_something} pour qu'elle respecte les spécifications précisées dans pour chaque script.

\begin{enumerate}
% 1
\item
\codeintext{do\_something} prend deux \codeintext{str} en arguments et retourne la \codeintext{str} avec le plus de caractères.
Si il y a égalité du nombre de caractères, elle retourne le premier mot.
\begin{lstlisting}

def do_something(           ):






word1 = 'Ryu'
word2 = 'Bison'

print(do_something(word1, word2)
\end{lstlisting}


\item
\codeintext{do\_something} prend une \codeintext{list} en argument.
Elle va trier la liste (méthode \codeintext{sort}) et renvoyer la valeur du milieu de la liste, donc celle qui est à l'index égal à la division entière de la longueur du tableau par 2.

\begin{lstlisting}

def do_something(           ):






l = [5,2,2,5,9]
median = do_something(l)
print(median)
\end{lstlisting}

\pagebreak
\item
\codeintext{do\_something} prend deux \codeintext{str} en arguments et retourne les trois premiers caractères de la première chaîne de caractères concaténés aux deux premiers caractères de la deuxième chaîne.
\begin{lstlisting}

def do_something(           ):





first_name = 'Luigi'
last_name = 'Mario'

sw_name =  do_something
print(sw_name)
\end{lstlisting}


\item
\codeintext{do\_something} prend deux arguments, une liste (\codeintext{list}) et un entier (\codeintext{int}).
Elle va parcourir la liste et imprimer chaque élément tant qu'elle ne tombe pas sur le deuxième argument.
Notez que cette fonction ne renvoie rien.
\begin{lstlisting}

def do_something(           ):




do_something([1, 2, 3, 4, 6], 3)
\end{lstlisting}
\end{enumerate}




\end{document}
